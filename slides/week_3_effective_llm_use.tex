\documentclass[aspectratio=169]{beamer}
\usepackage{beamerthemeAMU}
\usepackage{amsmath}
\usepackage{listings}
\usepackage{booktabs}
\usepackage[T1]{fontenc}
\usepackage[utf8]{inputenc}

\lstset{basicstyle=\ttfamily\small,keywordstyle=\color{AMU_dk2}\bfseries,commentstyle=\itshape\color{gray},showstringspaces=false,columns=flexible}

\title{Large Language Models in Data Science}
\subtitle{Week 3: Prompting for Effective LLM Use}
\author{Sebastian Mueller}
\institute{Aix-Marseille Universit\'e}
\date{2025-2026}

\begin{document}

\begin{frame}[plain]
  \titlepage
\end{frame}

\begin{frame}{Session Overview}
  \begin{columns}[T,onlytextwidth]
    \begin{column}{0.55\linewidth}
      \textbf{Lecture (1.5h)}
      \begin{enumerate}
        \item Clear rules for prompting
        \item Roles: system vs. user (with example)
        \item Separate data from instructions
        \item Format-controlled outputs (JSON/code)
        \item Two-step prompting (prompt-writer)
        \item Coding, research, ideation patterns
        \item Risks, guardrails, references
      \end{enumerate}
    \end{column}
    \begin{column}{0.4\linewidth}
      \textbf{Lab (1.5h)}
      \begin{itemize}
        \item Gemini 1.5 Flash in notebooks
        \item Role prompting: see outcome change
        \item Structured prompts and JSON parsing
        \item Generate and run small EDA code
        \item Optional: search+summarize scaffold
      \end{itemize}
    \end{column}
  \end{columns}
\end{frame}

\section{Why Prompting Matters}

\begin{frame}{Prompting Makes a Big Difference}
  \begin{itemize}
    \item Clarity and structure often outweigh model choice for many tasks.
    \item A clear goal and acceptance criteria lead to better outputs.
    \item Save prompts and parseable outputs to reproduce analyses.
  \end{itemize}
\end{frame}


\section{Prompting}

\begin{frame}{Clear Rules for Prompting}
  \begin{itemize}
    \item \textbf{Be specific}: Ask exactly for what you need; set scope and length.
    \item \textbf{State role}: Use a system instruction to set role and guardrails.
    \item \textbf{Separate parts}: Keep instructions, data, and output format distinct.
    \item \textbf{Format outputs}: Require JSON or fenced code for parsing.
    \item \textbf{Acceptance criteria}: Provide a checklist; ask the model to self-verify.
    \item \textbf{Few-shot (minimal)}: One short example can anchor style reliably.
  \end{itemize}
\end{frame}

\begin{frame}[fragile]{Reusable Prompt Template}
\begin{lstlisting}[language=Markdown]
System: You are a senior data scientist.
Prefer minimal, correct code. Follow the output format exactly.

User:
- Task: <what to do, be concrete>
- Context: <one paragraph; optional>
- Constraints: <libraries, time, memory, safety>
- Output format: <strict JSON schema or sections>
- Self-check: Confirm acceptance criteria before finalizing.
\end{lstlisting}
\end{frame}

\section{Roles}

\begin{frame}{Roles: System vs. User}
  \begin{itemize}
    \item \textbf{System instruction}: Persistent role and guardrails (e.g., "You are a cat.").
    \item \textbf{User prompt}: The immediate task/question.
    \item Changing the system instruction can change style, persona, and assumptions.
  \end{itemize}
\end{frame}

\begin{frame}[fragile]{Role Prompting: Minimal Example}
\begin{lstlisting}[language=Python]
import os, google.generativeai as genai
genai.configure(api_key=os.getenv("GEMINI_API_KEY"))

PROMPT = "In one sentence, what do you think about mice?"

# No system instruction
plain = genai.GenerativeModel("gemini-1.5-flash")
print(plain.generate_content(PROMPT).text)

# System instruction alters tone/persona
cat = genai.GenerativeModel("gemini-1.5-flash",
    system_instruction="You are a cat.")
print(cat.generate_content(PROMPT).text)
\end{lstlisting}
\end{frame}

\section{Structure}

\begin{frame}{Separate Data from Instructions}
  \begin{itemize}
    \item Keep instructions stable; swap data without changing behavior.
    \item Use delimiters or fields: \texttt{Instructions: ...} \quad \texttt{Data: <...>}
    \item Benefits: auditability, reproducibility, safer copy/paste.
  \end{itemize}
\end{frame}

\begin{frame}[fragile]{Formatting Outputs for Parsing}
\begin{lstlisting}[language=Markdown]
System: Return strict JSON with keys {"summary": str, "tags": [str]}.
User:
- Instructions: Summarize the text and produce 3 topical tags.
- Data: <article text here>
\end{lstlisting}
\vspace{0.4em}
\textit{Tip:} Validate with \texttt{json.loads} and ask the model to fix on failure.
\end{frame}

\section{Patterns}

\begin{frame}{Two-Step Prompting (Prompt-Writer)}
  \begin{itemize}
    \item Step 1: Ask the LLM to draft/refine the prompt for your task.
    \item Step 2: Use the drafted prompt (after personal refining) to run the task (same or different model).
    \item Useful for clarifying objectives and output format before execution.
  \end{itemize}
\end{frame}

\begin{frame}{Code, Research, Ideation: Quick Patterns}
  \begin{itemize}
    \item \textbf{Code}: generate \(\rightarrow\) run \(\rightarrow\) feed results back; ask for tests.
    \item \textbf{Research}: generate queries \(\rightarrow\) gather snippets \(\rightarrow\) synthesize after verification and add citations.
    \item \textbf{Ideation}: require 3 distinct strategies with trade-offs and a next step.
  \end{itemize}
\end{frame}


\section{Risks}

\begin{frame}{Risks and Guardrails}
  \begin{itemize}
    \item \textbf{Correctness}: Prefer code+execution; verify with tests.
    \item \textbf{Safety}: Sandbox code; avoid secrets; pin deps.
    \item \textbf{Reproducibility}: Save prompts, model names, seeds, outputs.
  \end{itemize}
\end{frame}

\begin{frame}{References}
  \begin{itemize}
    \item Anthropic: Prompt Engineering Interactive Tutorial (thinking step-by-step)
    \item Anthropic: prompt-eng-interactive-tutorial/06\_Precognition\_Thinking\_Step\_by\_Step.ipynb
    \item OpenAI: Prompt Engineering Guide
    \item DeepLearning.AI: Prompt Engineering short course
    \item Google: Prompting with Gemini (developers site)
    \item Microsoft: Prompt Engineering Guidelines
  \end{itemize}
\end{frame}

\begin{frame}{Takeaways}
  \begin{itemize}
    \item Clear, structured prompts + role control drive quality.
    \item Separate instructions/data; require parseable outputs.
    \item Iterate with two-step prompting before running code.
  \end{itemize}
\end{frame}

\end{document}

